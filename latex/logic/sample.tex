%%%%%%%%%%%%%%%%%%%%%%%%%%%%%%%%%%%%%%%%%%%%%%%%%%%%%%%%%%%%%%%%%%%%%%%%%%%%%%%%%%%%%
% To compile this file, run the following command (under Linux):
% > latex sample.tex
% Then, to view it, use the following command:
% > xdvi sample.dvi & 
% You can generate a PDF file (for printing etc) using the following command:
% > pdflatex sample.tex
% Use a program like emacs or xemacs to manipulate the tex-file, e.g.:
% > xemacs sample.tex &
%%%%%%%%%%%%%%%%%%%%%%%%%%%%%%%%%%%%%%%%%%%%%%%%%%%%%%%%%%%%%%%%%%%%%%%%%%%%%%%%%%%%%


\documentclass[11pt]{article}

\title{\textbf{\LaTeX\ for Logic}}
\author{Ulle Endriss}

\begin{document}
\maketitle


\section{Introduction}

We give some simple examples to demonstrate how you can use \LaTeX\ to produce 
a nice-looking document on a logic-related subject (or any scientific subject
in general). So this is about learning \LaTeX;\footnote{From now on I will 
simply write LaTeX rather than \LaTeX. It would just get a bit tedious otherwise.} 
don't take the actual content too serious.

Section~\ref{logicexamples}, for instance, provides some logic-related examples. 
(At the time of writing this, I didn't know which number that section would 
get in the end, so I used a combination of the \texttt{label}- and the
\texttt{ref}-command; see source file \texttt{sample.tex}.)

% This is a comment, i.e. this text will not appear in the dvi-file.


\section{Basic Stuff}

The text is written in a simple text editor and then formatted using the 
\texttt{latex}-command. Spacing, page breaks, numbering of sections, subsections,
definitions, and other environments are all done automatically. This helps a lot
when you develop large documents. Sometimes some ``manual'' adjustments are still
necessary---nothing is perfect---but these should be kept to an absolute minimum.
The underlying philosophy is that you concentrate on writing and leave the
design of the document to the professionals (that is, the people who developed 
LaTeX in this case).

\subsection{Fonts}

Look at the source for this paragraph to find out how to \emph{emphasise} parts
of a sentence or how to get LaTeX to display something in \textbf{bold font}.
In some case you may wish to use \texttt{typewriter font} or \textsf{sans serif}.
Maybe even \textsc{Capitalisation}, but for aesthetic reason you should probably
use all of these with some care.

\subsection{Mathematics}

If you want to write something mathematical, you need to switch to ``math mode''.
For something short within a paragraph, such as $f(x^2 + 17)$, use dollar signs. 
For a bigger formula that you want to be centred you could, for example, use the
\texttt{eqnarray}-environment:
\begin{eqnarray}
\sum_{i=1}^n i & = & \frac{n \cdot (n+1)}{2}
\end{eqnarray}
% If you use eqnarray* instead you won't get an ``equation number''.

Please note that this is just one of several ways of doing something like this.
You can find other examples throughout this document.

There are special commands for all sorts of mathematical symbols. Here are just
some of them: $\alpha$, $\beta$, $\gamma$, $\subseteq$, $\equiv$, $\approx$,
$\Leftarrow$, $\leftrightarrow$, $\infty$, $\in$, $\leq$, and so on \ldots 


\section{Some Logic Examples}\label{logicexamples}

Here's a list of examples:
\begin{enumerate}
\item $(\neg A \vee B) \leftrightarrow (A \rightarrow B)$
\item $(\forall x) P(x) \wedge (\exists x) Q(x) \rightarrow 
  (\exists x)(P(x) \wedge Q(x))$
\item ${\cal M} \models \varphi \vee (\psi_1 \wedge \psi_2)$
\item ${\cal M} \not\models \neg \varphi$
\item
  % We use \mbox to switch back to text mode within math mode and then 
  % \textit to get the words in italics:
  $\mbox{\textit{Famous}} \sqcap
  \exists \mbox{\textit{dislikes}}.(\mbox{\textit{Rich}} \sqcup
  \neg \mbox{\textit{Talented}})$
\item $B \cup \{\varphi\}$ is satisfiable
\item $\{(x,y) \in {\cal D}^2 \;|\; x < y\}$
\item $\mu := \mu \circ [y\leftarrow g(b,a)]$
\item $(\forall x_1)\cdots(\forall x_n)\varphi\,[y\leftarrow f(x_1,\ldots,x_n)]$
\item $x_1$
\item $\top$ and $\bot$
\end{enumerate}

In LaTeX, there are two ways of writing the Greek letter \emph{phi}
(lowercase): $\phi$ and $\varphi$. I prefer the latter.


\section{More Examples}

And some more examples \ldots

\subsection{Program Code}

You can use the \texttt{verbatim}-environment to typeset short programs. It 
tells LaTeX to print everything exactly as it appears in the source file, using 
typewriter font (that is, the real LaTeX is essentially ``switched off'').
\begin{quote} % used to get the indentation
\begin{verbatim}
reverse_list( [], []).

reverse_list( [Head | Tail], ReversedList) :-
  reverse_list( Tail, ReversedTail),
  append( ReversedTail, [Head], ReversedList).
\end{verbatim}
\end{quote}

\subsection{Tables}

Here's a simple table:
\begin{center}
\begin{tabular}{|l||c|c|c|}
\hline
Person & Biblical? & Female? & Superstar? \\
\hline\hline
Adam      & yes & no  & no  \\
Britney   & no  & yes & yes \\
Christina & no  & yes & yes \\
Eve       & yes & yes & no  \\
\hline
\end{tabular}
\end{center}


\section{Getting More Information}

There's a lot of information available on the web. 
Try these links:
\begin{itemize}
\item \texttt{http://www.latex-project.org/intro.html}
\item \texttt{http://www.ctan.org/tex-archive/info/lshort/english/}
\end{itemize}


\end{document}
