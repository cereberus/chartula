\documentclass{article}

\usepackage{mdframed}
\usepackage{alltt}

\usepackage{graphicx}

\usepackage{polski}
\usepackage[utf8]{inputenc}

\usepackage{listings}
\usepackage{csvsimple}


\begin{document}

\title{Sprawozdanie z zadania nr 9}
\author{Mikołaj Buchwald}

\maketitle
% % %

\begin{abstract}
    W ninejszej pracy zaprezentowano rozwiązanie do zadania nr 9 z labolatoriów z programowania. Zbiór danych uzyskano przeprowadzając krótkie badanie z użyciem MindWave Mobile. Za pomocą programu napisanego w języku python zebrano surowy odczyt z jednoelektrodowego elektroencefalografu. Zbadano również poziom atencji oraz medytacji osoby badanej. Dane zostały zaprezentowane z użyciem programu scilab na trzech wykresach. Dokonano również podstawowych obliczeń statystycznych na zebranym materiale. 
\end{abstract}
% % % % %

\newpage
\section{Opis zbioru danych}
Surowy sygnał eeg oraz czas jego pobrania. Cała tabela ma 7459 wierszy więc zamieszczam tylko ostatnie 14.
\newline
\begin{table}[h]
\begin{center}
\caption[Table caption text]{Sygnał eeg i czas jego uzyskania}
\csvautotabular{baseline.csv}
\end{center}
\end{table}

Czasy pobrania oraz poziomy odpowiednio atencji oraz medytacji.
\newline
\begin{table}[h]
\center \caption[Table caption text]{Poziomy atencji oraz medytacji i czas ich uzyskania}
\csvautotabular{esense.csv}
\end{table}
% % %
\newpage
\section{Trzy wykresy wraz z modyfikacjami}

\begin{figure}[htbp]
    \centering
    \includegraphics[width=\linewidth]{scilabwykres.jpg}
    \caption{Trzy podwykresy stworzone w Scilab}
\end{figure}

% % % % %

\section{Obliczone oraz wyświetlone w konsoli: średnia, odchylenie standardowe oraz wariancja}

Obliczenia wykonano dla sygnału eeg.
Poniżej podano wyniki z konsoli.
\newline
\begin{mdframed}
\begin{alltt}
signal\_mean   
 
    1.9177947  
 
    
 
 signal\_stdev   
 
    2.2468448  
 
    
 
signal\_variance   
 
    5.0483117  
 
\end{alltt}
\end{mdframed}

% % %
\section{Obliczona oraz wyświetlona w konsoli korelacja}

Korelacja sygnału z czasem.
Poniżej podano wyniki z konsoli.
\newline
\begin{mdframed}
\begin{alltt}
signal\_correl (with time)   
 
  - 0.0163650  
\end{alltt}
\end{mdframed}

% % %

\section{Odpowiedzi na pytania}

Odpowiadzi na pytania:
\begin{itemize}
    \item{Jaki jest średni sygnał pomiędzy czwartą a ósmą sekundą?}
    \item{Czy wyższy jest średni  poziom atencji czy medytacji?}
\end{itemize}
Poniżej podano wyniki z konsoli.
\newline

\begin{mdframed}
\begin{alltt}
 What is mean signal value between 4th and 8th second?   
 
 signal\_mean\_range in range 4-8 seconds   
 
    2.3196808  
 
    
 
 Is the mean attention higher or the mean meditation?   
 
 Mean attention is higher with the value:   
 
    35.375  
 
 While mean meditation is:   
 
    16.625   
\end{alltt}
\end{mdframed}

% % %

\section{Skrypt wraz z komentarzem}

\begin{mdframed}
\lstinputlisting{eeg_mwm.sce}
\end{mdframed}
% % %

\end{document}
